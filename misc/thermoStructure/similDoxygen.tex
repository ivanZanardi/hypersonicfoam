\documentclass[a4paper,11pt]{article}
\usepackage[latin1]{inputenc}
\newcommand{\scriptword}[1]{\textsf{\ttfamily #1}}
\newcommand{\classword}[1]{\textit{\textsf{\sffamily #1}}}


\usepackage{tikz}
\usetikzlibrary{shapes,arrows}
\usetikzlibrary{intersections}
\tikzset{%
	% Specifications for style of nodes:
	base/.style = {rectangle, rounded corners, draw=black, minimum width=3cm, 
		minimum height=2em, text centered},
	specieDir/.style = {base, minimum width=4.5cm, fill=blue!30,font=\ttfamily},
	basicDir/.style = {base, minimum width=4.5cm, fill=red!30,font=\ttfamily},
	reactionThermoDir/.style = {base, minimum width=4.5cm, fill=green!30,font=\ttfamily},
}
\tikzstyle{line} = [draw, -latex']


\usepackage{titling}
\setlength{\droptitle}{-12em}
\title{OpenFOAM \scriptword{thermophysicalModels} structure}
\author{Ivan Zanardi}
\date{}



\begin{document}

\maketitle

%\documentclass[border=10pt]{standalone} --> in separated document
\rotatebox{90}{
	\scalebox{0.784}[0.784]{
\begin{tikzpicture}[node distance=2cm,every node/.style={fill=white,
	font=\sffamily}, align=center]
% Place nodes
% Group on the left
\node [base] (tT) {thermoType};
\node [base, right of=tT, xshift=3cm, yshift=3cm] (type) {\textit{type}};
\node [base, right of=tT, xshift=3cm, yshift=2cm] (mixture) {\textit{mixture}};
\node [base, right of=tT, xshift=3cm, yshift=1cm] (transport) {\textit{transport}};
\node [base, right of=tT, xshift=3cm] (thermo) {\textit{thermo}};
\node [base, right of=tT, xshift=3cm, yshift=-1cm] (energy) {\textit{energy}};
\node [base, right of=tT, xshift=3cm, yshift=-2cm] (EoS) {\textit{equationOfState}};
\node [base, right of=tT, xshift=3cm, yshift=-3cm] (specie) {\textit{specie}};
\node [basicDir, right of=type, xshift=3cm] (hPT) {hePsiThermo};
\node [reactionThermoDir, right of=mixture, xshift=3cm] (mCM) {multiComponentMixture};
\node [specieDir, right of=transport, xshift=3cm] (sutherland) {sutherland};
\node [specieDir, right of=thermo, xshift=3cm] (janaf) {janaf};
\node [specieDir, right of=energy, xshift=3cm] (sIE) {sensibleInternalEnergy};
\node [specieDir, right of=EoS, xshift=3cm] (pG) {perfectGas};
\node [specieDir, right of=specie, xshift=3cm] (sp) {specie};
% Group at the top-right
\node [basicDir, right of=hPT, xshift=3cm, yshift=1.5cm, minimum width=2.5cm] (pT) {psiThermo};
\node [basicDir, right of=hPT, xshift=3cm, yshift=-1cm, minimum width=2.5cm] (hT) {heThermo};
\node [basicDir, right of=pT, xshift=2cm, minimum width=3cm] (fT) {fluidThermo};
\node [basicDir, right of=fT, xshift=1.5cm, minimum width=2.5cm] (bT) {basicThermo};
\node [basicDir, below of=fT, yshift=0.5cm, minimum width=3cm] (pTs) {"psiThermos.C"};
\node [basicDir, below of=bT, yshift=-0.5cm, minimum width=2.5cm] (pM) {pureMixture};
\node [basicDir, below of=pTs, yshift=-0.25cm, minimum width=2.5cm] (bMnew) {basicMixture};
\node [reactionThermoDir, above of=pT, yshift=4.5cm] (pRT) {psiReactionThermo};
\node [reactionThermoDir, below of=pRT, xshift=6cm] (pRTs) {"psiReactionThermos.C"};
\node [specieDir, below of=pRTs, xshift=2cm, yshift=-0.25cm, minimum width=2.5cm] (t) {thermo};
\node [reactionThermoDir, below of=pRTs, xshift=-2.5cm, yshift=0.5cm] (mRT) {"makeReactionThermo.H"};
\node [reactionThermoDir, below of=mRT, yshift=0.5cm, minimum width=2.5cm] (SM) {SpecieMixture};
\node [reactionThermoDir, below of=pRT, xshift=-2.25cm, minimum width=2.5cm] (bSM) {basicSpecieMixture};
\node [reactionThermoDir, left of=bSM, xshift=-1.5cm, yshift=-1.5cm, minimum width=2.5cm] (bMCM) {basicMultiComponentMixture};
\node [basicDir, below of=bMCM, yshift=0.5cm, minimum width=2.5cm] (bM) {basicMixture};
% Group at the top-left = legend
\node [specieDir, above of=tT, xshift=-1cm, yshift=9.5cm, minimum width=0.8cm] (specieLeg) {};
\node [basicDir, below of=specieLeg, yshift=1cm, minimum width=0.8cm] (basicLeg) {};
\node [reactionThermoDir, below of=basicLeg, yshift=1cm, minimum width=0.8cm] (reactionLeg) {};
\node [base, right of=specieLeg, xshift=3cm, minimum width=7cm] (specieDir) {src/thermophysicalModels/specie/};
\node [base, right of=basicLeg, xshift=3cm, minimum width=7cm] (basicDir) {src/thermophysicalModels/basic/};
\node [base, right of=reactionLeg, xshift=3cm, minimum width=7cm] (reactionDir) {src/thermophysicalModels/reactionThermo/};
% Draw edges
% Group on the left
\path [line,dashed] (tT.east) -- ++(0.5,0) |- (type.west);
\path [line,dashed] (tT.east) -- ++(0.5,0) |- (mixture.west);
\path [line,dashed] (tT.east) -- ++(0.5,0) |- (transport.west);
\path [line,dashed] (tT.east) -- ++(0.5,0) |- (thermo.west);
\path [line,dashed] (tT.east) -- ++(0.5,0) |- (energy.west);
\path [line,dashed] (tT.east) -- ++(0.5,0) |- (EoS.west);
\path [line,dashed] (tT.east) -- ++(0.5,0) |- (specie.west);
%
\path [line,dashed] (type) -- (hPT);
\path [line,dashed] (mixture) -- (mCM);
\path [line,dashed] (transport) -- (sutherland);
\path [line,dashed] (thermo) -- (janaf);
\path [line,dashed] (energy) -- (sIE);
\path [line,dashed] (EoS) -- (pG);
\path [line,dashed] (specie) -- (sp);
%
\path [line] (sutherland.east) -- ++(1,0) |- (sp.east);
\path [line] (janaf.east) -- ++(1,0) |- (sp.east);
%
% Group at the top-right
\path [line] (hPT.east) -- ++(1,0) |- (pT.west);
\path [line] (hPT.east) -- ++(1,0) |- (hT.west);
\path [line] (pT.east) -- (fT.west);
\path [line] (fT) -- (bT);
\path [line] (pT.east) --  ++(0.5,0) |- (pTs.west);
\path [line] (pTs) --  ++(5.75,0) |- (t);
\path [line] (pTs) -| node[anchor=south] {calls also} (pM);
\path [line] (hT) -| (bMnew);
\path [line] (pM) -| (bMnew);
%
\path [line, draw=blue] (pRT) -- (pT);
\path [line, draw=blue] (pRT) -- ++(0,-1) -| (pRTs);
\path [line, draw=blue] (mRT) -- node[anchor=east] {calls} (SM);
\path [line, draw=blue] (pRTs) -- ++(0,-0.75) -| node[anchor=east] {includes} (mRT);
\path [line, draw=blue] (pRTs) -- ++(0,-0.75) -| node[anchor=west] {calls also} (t);
%
\path [line, draw=blue] (pRT) -- ++(0,-1) -| (bSM);
\path [line] (bSM) -| (bMCM);
\path [line] (bMCM) -- (bM);
%
\path [line] (mCM.east) -| (bSM);
%
% Group at the top-left = legend
\path [line,dashed] (specieLeg) -- node[anchor=south] {in} (specieDir);
\path [line,dashed] (basicLeg) -- node[anchor=south] {in} (basicDir);
\path [line,dashed] (reactionLeg) -- node[anchor=south] {in} (reactionDir);
%
\end{tikzpicture}
}}

Thermophysical models are concerned with energy, heat and physical properties. There is one compulsory dictionary entry called \scriptword{thermoType} which specifies the package of thermophysical modelling that is used in the simulation. The keyword entries in \scriptword{thermoType} reflects the multiple layers of modelling and the underlying framework in which they combined.
\newline

The most general solvers that use thermophysical modelling construct a \scriptword{fluidThermo} model that allows the user to specify the thermophysical model through the type entry at run-time.
\newline

Each solver that uses thermophysical modelling constructs an object of a specific thermophysical model class and the one used here is \scriptword{psiReaction\-Thermo}, constructed for reacting mixture and based on compressibility.

\section*{The \scriptword{thermoType} working flow}

\begin{itemize}
	
\item \classword{specie} $\rightarrow$ \scriptword{specie}: base class of the thermophysical property types.

There is currently only one option for the \scriptword{specie} model which specifies the composition of each constituent in terms of mass fraction \scriptword{Y} and molecular weight \scriptword{molWeight}.

\textbf{Modifications}: add the other species properties in order to make possible the definition of the vibro-electronic model, like
\begin{itemize}
	\item the characteristic vibrational temperature;
	\item the characteristic rotational temperature;
	\item the ionization energy;
	\item the dissociation potential.\\
\end{itemize}



\item \classword{equationOfState} $\rightarrow$ \scriptword{perfectGas}: class for perfect gas equation of state.

\textbf{Modifications}: change the temperature variable from \scriptword{T} to \scriptword{Ttr} (trans\--ro\-ta\-tio\-nal temperature), except for electrons for which \scriptword{psi} is a function of \scriptword{Tve} (vibro-electronic temperature).\\



\item \classword{energy} $\rightarrow$ \scriptword{sensibleInternalEnergy}: thermodynamics mapping class to expose the sensible internal energy functions.

The user must specify the form of energy to be used in the solution, sensible internal energy $e$ or sensible enthalpy $h$ (\scriptword{sensibleEnthalpy}).

\textbf{Modifications}: add the functions that consider also the energy forms and the heat capacities for the trans-rotational and vi\-bro-e\-lec\-tro\-nic mode.
\newline



\item \classword{thermo}

$\rightarrow$ \scriptword{janaf}: thermodynamic model that evaluates the specific heat and the other fundamental thermo-properties, from which the other ones are derived.

\textbf{Modifications}: construct the new thermo-model in order to evaluate the heat capacities and the other fundamental thermo-properties for the trans-rotational and vibro-electronic mode as a function of \scriptword{Ttr} and \scriptword{Tve}. \newline

$\rightarrow$ \scriptword{thermo}: basic thermodynamics type based on the use of fitting functions for $c_p$, $h$, $s$ obtained from the template argument type \scriptword{thermo$\rightarrow$ janaf} in order to obtain the temperature values by using the Newton-Raphson method
$$T_{n+1}=T_n-\frac{he(T_n)-he_{transEq}}{C_{pv}(T_n)}$$
until convergence ($T_{n+1}\approx T_n$). All other properties are derived from these primitive functions.

\textbf{Modifications}: construct the new thermo-model in order to evaluate the heat capacities and the other derived thermo-properties for the trans-rotational and vibro-electronic mode as a function of \scriptword{Ttr} and \scriptword{Tve}, calculated using the method above. \underline{Pay attention to temperatu-} \underline{res treatment}.
\newline



\item \classword{transport} $\rightarrow$ \scriptword{sutherland}: transport package using Sutherland's formula, templated into a given thermodynamics package (needed for thermal conductivity). It calculates $\mu$ as a function of temperature $T$ from a Sutherland coefficient \scriptword{As} and Sutherland temperature \scriptword{Ts}.

The transport modelling concerns evaluating dynamic viscosity $\mu$, thermal conductivity $\kappa$ and thermal diffusivity $\alpha$ (for internal energy and enthalpy equations).

\textbf{Modifications}: modify the expressions for the transport properties in order to fit the two-temperature model and add the new ones used in the vibro-electronic conservation equation.
\newline



\item \classword{mixture}
\begin{enumerate}
	\item $\rightarrow$ \scriptword{basicMixture}: base class of the mixture. \newline
	
	$\rightarrow$ \scriptword{pureMixture}: mixture with fixed composition. \newline
	
	\item $\rightarrow$ \scriptword{basicMixture}: base class of the mixture. \newline
	
	$\rightarrow$ \scriptword{basicMultiComponentMixture}: mixtures with different components. Provides a list of mass fraction fields and helper functions to query mixture composition. \newline
	
	$\rightarrow$ \scriptword{basicSpecieMixture}: specialization of \scriptword{basicMultiCompo\-nentMixture} for a mixture consisting of a number for molecular species. Initialization of per specie thermo/transport properties and specific specie properties. \newline
	
	$\rightarrow$ \scriptword{multiComponentMixture}: mixtures with variable composition, required by thermophysical models with reactions. It defines the functions for mixture composition calculations within the cell and at the boundary faces, through which the mixture thermo-properties can be evaluated. \newline
	
	\item $\rightarrow$ \scriptword{SpecieMixture}: specialization of \scriptword{basicMultiComponentMix\-ture} for a mixture consisting of a number for molecular species. Calling of the effective per specie thermo/transport properties and specific specie properties. \newline
\end{enumerate}

\textbf{Modifications}: add per specie thermo/transport properties and specific specie properties so that allow the definitions of the two-tempe\-ra\-tu\-re model.
\newline



\item \classword{type}

$\rightarrow$ \scriptword{basicThermo}: abstract base-class for fluid and solid thermodynamic properties. It performs different tasks:
\begin{itemize}
	\item it defines the function that returns the enthalpy/internal energy field boundary types by interrogating the temperature field boundary types  \scriptword{heBo\-undaryTypes()};
	\item it checks that the thermodynamics package is consistent with energy forms supported by the application;
	\item it initializes of the update properties function \scriptword{correct()};
	\item it initializes the access to the thermodynamic/transport state and derived variables.\\ 
\end{itemize}

$\rightarrow$ \scriptword{fluidThermo}: initialization of fundamental fluid thermodynamic properties like $\mu$, $\psi$ and the $\rho$ correction \scriptword{correctRho} (i.e. the correction of the density field used to update the density field following the pressure solution) and the definition of $\nu$ from $\mu$ and $\rho$. \\

$\rightarrow$ \scriptword{psiThermo}: basic thermodynamic properties based on compressibility. It's characterized by the following features:
\begin{itemize}
	\item the density correction \scriptword{correctRho()} gives nothing;
	\item the density calculation \scriptword{rho()} is performed using the current value of pressure by applying the perfect gas equation of state.\\ 
\end{itemize}

$\rightarrow$ \scriptword{heThermo}: enthalpy/internal energy values for a mixture. It performs different tasks:
\begin{itemize}
	\item it returns the composition of the mixture;
	\item it accesses to thermodynamic state variables \scriptword{he};
	\item it returns the fields derived from thermodynamic/transport state variables like temperatures and heat capacities or thermal diffusivity.\\ 
\end{itemize}

$\rightarrow$ \scriptword{hePsiThermo}: Energy for a mixture based on compressibility. It performs the following tasks:
\begin{itemize}
	\item it calculate the thermo variables of the entire mixture (temperature within the cells and at the patch faces and the other properties that derive from $T$) through the \scriptword{calculate()} function;
	\item it updates properties through the \scriptword{correct()} function by saving the old-time values.\\ 
\end{itemize}

\textbf{Modifications}: by follwing the \classword{type} working flow, add the thermodynamic/transport state and derived variables so that allow the definitions of the two-tempe\-ra\-tu\-re model.

\end{itemize}
	
\end{document}

